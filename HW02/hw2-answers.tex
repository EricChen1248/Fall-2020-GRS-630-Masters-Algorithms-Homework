\documentclass[a4paper,12pt]{article}
\usepackage[
  top=1.8cm,
  bottom=1.5cm,
  left=1.8cm,
  right=1.8cm,
  includefoot,
  includehead]{geometry}

\usepackage[utf8]{inputenc}
\usepackage{changepage}
\usepackage{multicol}
\usepackage{nccmath}
\usepackage{amsmath}
\usepackage{calc}
\usepackage{enumitem}

% For fancy math
\RequirePackage{amsmath,amsthm,amssymb}
\newtheorem{theorem}{Theorem}
\newtheorem{fact}[theorem]{Fact}
\newtheorem{lemma}[theorem]{Lemma}
\newtheorem{claim}[theorem]{Claim}

\newcommand{\ord}[2][th]{\ensuremath{{#2}^{\mathrm{#1}}}}
% shorthand for \mathcal{O}
\newcommand{\Ocal}{\ensuremath{\mathcal{O}}}
\newcommand{\aug}{\fboxsep=-\fboxrule\!\!\!\fbox{\strut}\!\!\!}
\newcommand{\contradiction}{%
  \ensuremath{{\Rightarrow\mspace{-2mu}\Leftarrow}}%
}

% Counters for HW number, author, and collaborators
\newcommand{\hwnumber}[1]{\def\hwnumberdata{#1}}
\def\hwnumberdata{\relax}
\renewcommand{\author}[1]{\def\authordata{#1}}
\def\authordata{\relax}
\newcommand{\collaborators}[1]{\def\collaboratorsdata{#1}}
\def\collaboratorsdata{\relax}

% Fancy headings
\RequirePackage{fancyhdr}
\pagestyle{fancyplain}

\fancyhead[L]{\small \authordata \\
\small CS 630 Homework \#\hwnumberdata \\
  \textsl{Collaborators}: \collaboratorsdata}

\RequirePackage{titlesec}
\titleformat{\subsection}{\normalsize\bfseries}{\thesubsection}{.5em}{}
\renewcommand{\thesubsection}{\alph{subsection})}

% Making the problem and ppart environments
\newcommand{\addmedskip}{\addvspace{2\medskipamount}}
\newcommand{\addbigskip}{\addvspace{2\bigskipamount}}
\newcommand{\nline}{\bigskip}

\newcounter{problemnum}
\setcounter{problemnum}{0}
\newenvironment{problem}
  {\addbigskip \setcounter{partnum}{0}
   \noindent\stepcounter{problemnum}\textbf{Problem \arabic{problemnum}.\ }}
  {\par\addbigskip}

\newcounter{partnum}
\setcounter{partnum}{0}

\newenvironment{ppart}[1][]{%
  \addmedskip
  \refstepcounter{partnum}%\par\medskip%
  \enumerate[labelsep=*]\item[\textbf{\roman{partnum})}]}
  {\endenumerate}

\newsavebox{\mybox}
\newenvironment{answer}
{\begin{lrbox}{\mybox}\begin{minipage}{0.95\textwidth}\vspace{0.2cm}}
  {\vspace{0.1cm}\end{minipage}\end{lrbox}\fbox{\usebox{\mybox}}}

\newenvironment{customProof}{\begin{proof}\noindent}{\end{proof}}

% Put your name and the homework number here.
\author{Jiun-Yan (Eric) Chen}
\hwnumber{2}


\begin{document}
\vspace*{0.5\baselineskip}
\textbf{Please limit your answer to the following problems to at most 1/2 a page each.}

\collaborators{None} % Put your collaborators for the problem here
\begin{problem}
    LU decomposition for 2x2 singular matrices.

    \begin{ppart}
        \textbf{T} or \textbf{F}: The 2x2 matrix of all 0's has an LU decomposition. Show your LU decomposition or explain why there is none.
    \end{ppart}

    \begin{answer}
        $L = \begin{bmatrix}
            1 & 0 \\
            0 & 1
        \end{bmatrix}$, 
        $U = \begin{bmatrix}
            0 & 0 \\
            0 & 0
        \end{bmatrix}$ can form a valid LU decomposition for a zero matrix of size 2x2.
    \end{answer}

    \begin{ppart}
        \textbf{T} or \textbf{F}: The 2x2 matrix of all -1's has an LU decomposition. Show your LU decomposition or explain why there is none.
    \end{ppart}

    \begin{answer}
        $L = \begin{bmatrix}
            1 & 0 \\
            1 & 1
        \end{bmatrix}$, 
        $U = \begin{bmatrix}
            -1 & -1 \\
            0 & 0
        \end{bmatrix}$ can form a valid LU decomposition for a 2x2 matrix of all -1's.
    \end{answer}


    \begin{ppart}
        \textbf{T} or \textbf{F}: Every singular 2x2 Boolean matrix has an LU decomposition. Prove or give a counterexample. Note: Entries of a Boolean matrix are 0's or 1's
    \end{ppart}

    \begin{answer}
        \begin{proof}
            Suppose all singular 2x2 Boolean matrix has an LU decomposition.

        Let 
        $A = \begin{bmatrix}
            a_{1,1} & a_{1,2} \\
            a_{2,1} & a_{2,2}
        \end{bmatrix}$
        $L = \begin{bmatrix}
            1 & 0 \\
            l & 1
        \end{bmatrix}$, 
        $U = \begin{bmatrix}
            u_{1,1} & u_{1,2} \\
            0 & u_{2,2}
        \end{bmatrix}$, where $a_{i,j} \in \{0, 1\}$

        \nline
        We can construct the following equations:

        \begin{equation*}
            \begin{cases}
                a_{1,1} = 1 \times u_{1,1} & _{(1)}\\
                a_{1,2} = 1 \times u_{1,2} & _{(2)}\\
                a_{2,1} = u_{1,1} \times l & _{(3)}\\
                a_{2,2} = u_{1,2} \times l + 1 \times u_{2,2} & _{(4)}
            \end{cases}
        \end{equation*}

        From (1), we can obtain $u_{1,1} = a_{1,1}$, and we can insert it into (3) to get $a_{2,1} = a_{1,1} \times l$

        If $a_{1,1} = 0$ and $a_{2,1} = 1$, then the equation $a_{2,1} = a_{1,1} \times l => 1 = 0 \times l  \contradiction $ 

        $\therefore$ not all Boolean matrices have an LU decomposition.
    \end{proof}
    \end{answer}

\end{problem}

\newpage

\collaborators{None} % Put your collaborators the problem here
\begin{problem}
    Find the LUP decomposition of the following 4 by 4 matrix M. Show one or two steps of your work, enough to show you are following the LUP algorithm.

    \nline

    $\begin{bmatrix}
        2 & -1 & 2 & 0 \\
        2 & -1 & 0 & 3 \\
        4 & 2 & -1 & 1 \\
        0 & 1 & 1 & 0
    \end{bmatrix}$

    \nline

    \begin{answer}
        \begin{tabular}{p{0.3\textwidth} p{0.3\textwidth} p{0.3\textwidth}}
            $A = \begin{bmatrix}
                2 & -1 & 2 & 0 \\
                2 & -1 & 0 & 3 \\
                4 & 2 & -1 & 1 \\
                0 & 1 & 1 & 0
            \end{bmatrix}$
            & $L = \begin{bmatrix}
                1 & 0 & 0 & 0 \\
                & 1 & 0 & 0 \\
                &  & 1 & 0 \\
                &  &  & 1  
            \end{bmatrix}$
            & $P = \begin{bmatrix}
                &  &  &  \\
                &  &  &  \\
                &  &  &  \\
                &  &  &   
            \end{bmatrix}$\\\\
            $A = \begin{bmatrix}
                4 & 2 & -1 & 1 \\
                2 & -1 & 2 & 0 \\
                2 & -1 & 0 & 3 \\
                0 & 1 & 1 & 0
            \end{bmatrix}$
            & $L = \begin{bmatrix}
                1 & 0 & 0 & 0 \\
                & 1 & 0 & 0 \\
                &  & 1 & 0 \\
                &  &  & 1  
            \end{bmatrix}$
            & $P = \begin{bmatrix}
                0 & 0 & 1 & 0 \\
                &  &  &  \\
                &  &  &  \\
                &  &  &   
            \end{bmatrix}$\\\\
            $A = \begin{bmatrix}
                4 & 2 & -1 & 1 \\
                0 & -2 & \frac{1}{2} & \frac{5}{2} \\
                0 & -2 & \frac{5}{2} & -\frac{1}{2} \\
                0 & 1 & 1 & 0
            \end{bmatrix}$
            & $L = \begin{bmatrix}
                1 & 0 & 0 & 0 \\
                \frac{1}{2} & 1 & 0 & 0 \\
                \frac{1}{2} &  & 1 & 0 \\
                0 &  &  & 1  
            \end{bmatrix}$
            & $P = \begin{bmatrix}
                0 & 0 & 1 & 0 \\
                &  &  &  \\
                &  &  &  \\
                &  &  &   
            \end{bmatrix}$\\\\
            $A = \begin{bmatrix}
                4 & 2 & -1 & 1 \\
                0 & -2 & \frac{1}{2} & \frac{5}{2} \\
                0 & 0 & 2 & -3 \\
                0 & 0 & \frac{5}{4} & \frac{5}{4}
            \end{bmatrix}$
            & $L = \begin{bmatrix}
                1 & 0 & 0 & 0 \\
                \frac{1}{2} & 1 & 0 & 0 \\
                \frac{1}{2} & 1 & 1 & 0 \\
                0 & -\frac{1}{2} &  & 1  
            \end{bmatrix}$
            & $P = \begin{bmatrix}
                0 & 0 & 1 & 0 \\
                0 & 1 & 0 & 0 \\
                &  &  &  \\
                &  &  &   
            \end{bmatrix}$\\\\
            $U = \begin{bmatrix}
                4 & 2 & -1 & 1 \\
                0 & -2 & \frac{1}{2} & \frac{5}{2} \\
                0 & 0 & 2 & -3 \\
                0 & 0 & 0 & \frac{25}{8} 
            \end{bmatrix}$
            & $L = \begin{bmatrix}
                1 & 0 & 0 & 0 \\
                \frac{1}{2} & 1 & 0 & 0 \\
                \frac{1}{2} & 1 & 1 & 0 \\
                0 & -\frac{1}{2} & \frac{5}{8} & 1  
            \end{bmatrix}$
            & $P = \begin{bmatrix}
                0 & 0 & 1 & 0 \\
                0 & 1 & 0 & 0 \\
                1 & 0 & 0 & 0 \\
                0 & 0 & 0 & 1  
            \end{bmatrix}$
        \end{tabular}
    \end{answer}
\end{problem}

\newpage

\begin{problem}
    We saw in class that you can reduce “finding the inverse of M” to “solving the LUP decomposition of M,” by using the LUP results from problem 2. You do this by setting up a system of 4 equations in four unknowns using the LUP decomposition. You do this once for each of the columns b of the I4 matrix as the right hand side of Mx = b.
\end{problem}

\begin{ppart}
    Write down the four Mx=b’s that you have to solve for x in order to compute the inverse.
    However, you need not solve them for x, (except for what is asked in part ii. below).
\end{ppart}
\begin{answer}
    If we solve the following four equations

    \nline
    \begin{equation*}
        Ax_1 = \begin{bmatrix} 1 \\ 0 \\ 0 \\ 0 \end{bmatrix}
        Ax_2 = \begin{bmatrix} 0 \\ 1 \\ 0 \\ 0 \end{bmatrix}
        Ax_3 = \begin{bmatrix} 0 \\ 0 \\ 1 \\ 0 \end{bmatrix}
        Ax_4 = \begin{bmatrix} 0 \\ 0 \\ 0 \\ 1 \end{bmatrix}
    \end{equation*}
    \nline

    Then $A^{-1} = \begin{bmatrix} x_1 & x_2 & x_3 & x_4\\ \end{bmatrix}$

\end{answer}

\newpage

\begin{ppart}
    Find the solution of x in part i. for the b = the first column of I4.
\end{ppart}
\begin{answer}
    To solve, we first substitute in $ LUx = PAx = b$

    \begin{equation*}
        LY = P\begin{bmatrix} 1 \\ 0 \\ 0 \\ 0 \end{bmatrix}, Ux = Y
    \end{equation*}

        \begin{align*}
            LY = 
            \begin{bmatrix}
                1 & 0 & 0 & 0 \\
                \frac{1}{2} & 1 & 0 & 0 \\
                \frac{1}{2} & 1 & 1 & 0 \\
                0 & -\frac{1}{2} & \frac{5}{8} & 1
            \end{bmatrix} 
            \begin{bmatrix}
                y_1 \\ y_2 \\ y_3 \\ y_4
            \end{bmatrix} 
    &= \begin{bmatrix}
        0 & 0 & 1 & 0 \\
        0 & 1 & 0 & 0 \\
        1 & 0 & 0 & 0 \\
        0 & 0 & 0 & 1  
    \end{bmatrix} \begin{bmatrix} 1 \\ 0 \\ 0 \\ 0 \end{bmatrix} = Px\\
    \begin{bmatrix}
        y_1 \\
        \frac{1}{2} y_1 + y_2 \\
        \frac{1}{2} y_1 + y_2 + y_3 \\
        -\frac{1}{2} y_2 + \frac{5}{8} y_3 + y_4 \\
    \end{bmatrix}
    &= \begin{bmatrix}
        0 \\
        0 \\
        1 \\
        0
    \end{bmatrix}\\
    \begin{cases}
        y_1 = 0 \\
        y_2 = 0 \\
        y_3 = 1 \\
        y_4 = -\frac{5}{8}
    \end{cases}
\end{align*}

\noindent\makebox[\linewidth]{\rule{\textwidth}{0.4pt}}

\begin{align*}
    Ux = \begin{bmatrix}
        4 & 2 & -1 & 1 \\
        0 & -2 & \frac{1}{2} & \frac{5}{2} \\
        0 & 0 & 2 & -3 \\
        0 & 0 & 0 & \frac{25}{8} \\
    \end{bmatrix}x &= \begin{bmatrix}
        0 \\
        0 \\
        1 \\
        -\frac{8}{5} \\
    \end{bmatrix}\\
    \begin{bmatrix}
        4x_1 + 2x_2 - x_3 + x_4 \\
        -2x_2 + \frac{1}{2}x_3 + \frac{5}{2}x_4 \\
        2x_3 - 3x_4\\
        \frac{25}{8}x_4 
    \end{bmatrix} &= \begin{bmatrix}
        0 \\
        0 \\
        1 \\
        -\frac{5}{8} \\
    \end{bmatrix}\\
    \begin{cases}
        x_1 = \frac{1}{5} \\
        x_2 = -\frac{1}{5} \\
        x_3 = \frac{1}{5} \\
        x_4 = -\frac{1}{5}
    \end{cases}
\end{align*}
\end{answer}

\newpage

\begin{problem}

    You are given an algorithm $A$ which can compute the square $B \times B$ of an n by n matrix in $\Ocal(n^2\sqrt{n})$ steps

\end{problem}
\begin{ppart}
    Show how to use A to compute the product of any two n by n matrices M1 and M2, also in $\Ocal(n^2\sqrt{n})$
\end{ppart}

\begin{answer}
    Construct a matrix $P = \begin{bmatrix}
        M1 & M2 \\
        I & I
    \end{bmatrix}$

    Then $P^2$ would equal $\begin{bmatrix}
        M1^2 + M2 & M1M2 + M2 \\
        M1 + I & M2 + I
    \end{bmatrix}$.

    Taking the upper right n x n corner of the $P^2$ matrix as $C$, then $M1M2 = C - M2$
\end{answer}

\begin{ppart}
    Show that your algorithm in part i. also takes $\Ocal(n^2\sqrt{n})$
\end{ppart}

\begin{answer}
    The $P$ matrix is an 2n x 2n matrix, thus the construction takes $\Ocal(4n^2)$

    The calculation of $P^2$ with the given algorithm A is $\Ocal(4n^2 \sqrt{2n})$

    Taking the upper right corner $C$ is $\Ocal(n^2)$ 

    Calculating $M1M2 = C - M2$ is a $\Ocal(n^2)$

    Thus, the total complexity is $\Ocal(4n^2) + \Ocal(4n^2 \sqrt{2n})  + \Ocal(n^2) + \Ocal(n^2) = \Ocal(n^2 \sqrt{n})$

\end{answer}

\newpage

\begin{problem}

    \begin{ppart}
        In a betting game you play against the house by picking 3 numbers from 1 to 30. You do this with replacement, that is you can pick some number more than once. Then the lottery person(or machine) chooses three numbers at random, also from 1 to 30.

        If you pick all three numbers correctly you win \$3000. Otherwise, you win \$400 if you picke any 2 out of the 3 numbers correctly. The game costs \$20 to play.

        What is the expected value you earn when playing the game ? To do this define the relevant random variable and compute its expectation.

    \end{ppart}
    
    \begin{answer}
        Chance to get all 3 correct: $\frac{1}{30}^3$

        Chance to get 2 out of 3 correct: $\frac{1}{30}^2 \times \frac{29}{30} \times C^3_2 $

        Expected Value = $\frac{1}{30^3} \times \$3000 + \frac{29 \times 3}{30^3} \times \$400 - \$20 = \$(\frac{-2511}{135}$
    \end{answer}

    \begin{ppart}
        Assume you have 3 balls each a different color, red, blue and green. You put the balls in a box and without looking into the box you pick one of the balls at random and remember its color. Then the ball is put back into the box and do this 2 more times. So you might have seen only 1 color or 2 colors or 3 colors in the 3 picks.

        What is the expected number of the different colors of the balls you see in your 3 picks ? Explain how you found your answer.
    \end{ppart}

    \begin{answer}
        The color of the first ball is irrelevant, all probabilities are $\frac{1}{3}$ of the chance, but there are $3\times$ of colors to get it, so we just assume that color of the first ball is fixed.

        \nline

        All possible  combination, assuming the first ball is red is: \\RRR RGG RBB RRB RBR RRG RGR RBG RGB

        1 color: RRR
        2 color: RGG RBB RBR RRB RGR RRG
        3 color: RGB RBG

        Expected value = $(1 \times 1 + 2 \times 6 + 3 \times 2) \div 9 = \frac{19}{9}$
    \end{answer}
\end{problem}
\end{document}
