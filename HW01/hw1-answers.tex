\documentclass[a4paper,12pt]{article}
\usepackage[
  top=1.8cm,
  bottom=1.5cm,
  left=1.8cm,
  right=1.8cm,
  includefoot,
  includehead]{geometry}

\usepackage[utf8]{inputenc}
\usepackage{changepage}

% For fancy math
\RequirePackage{amsmath,amsthm,amssymb}
\newtheorem{theorem}{Theorem}
\newtheorem{fact}[theorem]{Fact}
\newtheorem{lemma}[theorem]{Lemma}
\newtheorem{claim}[theorem]{Claim}

\newcommand{\ord}[2][th]{\ensuremath{{#2}^{\mathrm{#1}}}}
% shorthand for \mathcal{O}
\newcommand{\Ocal}{\ensuremath{\mathcal{O}}}

% Counters for HW number, author, and collaborators
\newcommand{\hwnumber}[1]{\def\hwnumberdata{#1}}
\def\hwnumberdata{\relax}
\renewcommand{\author}[1]{\def\authordata{#1}}
\def\authordata{\relax}
\newcommand{\collaborators}[1]{\def\collaboratorsdata{#1}}
\def\collaboratorsdata{\relax}


% Fancy headings
\RequirePackage{fancyhdr}
\pagestyle{fancyplain}

\fancyhead[L]{\small \authordata \\
\small CS 630 Homework \#\hwnumberdata \\
  \textsl{Collaborators}: \collaboratorsdata}

\RequirePackage{titlesec}
\titleformat{\subsection}{\normalsize\bfseries}{\thesubsection}{.5em}{}
\renewcommand{\thesubsection}{\alph{subsection})}

% Making the problem and ppart environments
\newcommand{\addmedskip}{\addvspace{2\medskipamount}}
\newcommand{\addbigskip}{\addvspace{2\bigskipamount}}
\newcommand{\nline}{\bigskip}

\newcounter{problemnum}
\setcounter{problemnum}{0}
\newenvironment{problem}
  {\addbigskip \setcounter{partnum}{0}
   \noindent\stepcounter{problemnum}\textbf{Problem \arabic{problemnum}.\ }}
  {\par\addbigskip}

\newcounter{partnum}
\setcounter{partnum}{0}
\newenvironment{ppart}
  {\addmedskip
   \noindent\stepcounter{partnum}\textbf{\roman{partnum})}\ }
  {\par\addbigskip}

\newenvironment{answer}
  {\begin{adjustwidth}{\parindent}{}\setlength{\parindent}{0pt}}{\end{adjustwidth}}

% Put your name and the homework number here.
\author{Jiun-Yan (Eric) Chen}
\hwnumber{1}


\begin{document}
\vspace*{0.5\baselineskip}
\textbf{Please limit your answer to the following problems to at most 1/2 a page each.} 

\collaborators{None} % Put your collaborators for Problem 1 here.

\begin{problem}
    You are given an integer $t$ and a polynomial $p(x)$ of degree $n$ where all of the $n+1$ coefficient $a_i$ of $p$ are non-zero integers. You want to compute $p(t)$.
    
    \begin{ppart}
    How many integer additions and integer multiplications does it take to compute $p(t)$ the normal way, i.e. plugging in $t$ for $x$ wherever $x$ occurs in $p(x)$ and then computing each term $a_it^i$ from left to right and summing the values as you go?
    \end{ppart}
    
    \begin{answer}
      To resolve each $a_it^i$, there is a multiplication step $a_i \times t \times t \times ... \times t$ where there are $i$ $t$s
      
      So for each term, there is $i+1$ multiplication steps. As there is $i+1$ terms, the total multiplications count is:
      
      \begin{equation*}
        \sum^{i+1}_{n=1} n = \frac{i+1(i+2)}{2}
      \end{equation*}

      To sum up all terms in $p(t)$, there is $a_it^i + a_{i-1}t^{i-1} ... + a_0$, where there are $i$ additions, as there are total of $i+1$ terms.
    \end{answer}

    \begin{ppart}
    Now see if you can find a better method to compute $p(t)$, using fewer than $\Ocal(n^2)$ adds and multiplies. Describe your method and explain clearly how many arithmetic operations it uses.
    \end{ppart}

    \begin{answer}
      There are two methods that could work, the first one makes the assumption that the algorithm can implement some form of caching.
      
      If we could cache each $t^n$, then it would be a trivial $\Ocal(1)$ operation to calculate $t^{n+1}$. 
      
      Thus the multiplications for each term $a_nt^n$ is $a_n \times t^{n-1} \times t$, which is 2 operations, and the total multiplication operations is $n \times 2$.

      \nline

      If we could not cache $t^n$, then we can also use a more efficient method of obtaining the powers, namely:
      \[
        \begin{cases}
          t^n = t^{n/2} \times t^{n/2},& \text{when n is an even number}\\
          t^n = t^{(n-1)/2} \times t^{(n-1)/2} \times t, & \text{when n is an odd number}
        \end{cases}
      \]
      With this method, we would have a minimum of $\log{n}$ operations, and a maximum of $2 \times \log{n}$ for each term. Summed up, this is equal to $n \log{n}$

      \nline

      For additions, as we will always have $n + 1$ terms, we will always need to perform $n$ addition operations.

      \nline

      Thus the total complexity for calculating a polynomial is $\Ocal(2n + n) = \Ocal(n)$ if caching is allowed, or $\Ocal(n\log{n} + n) = \Ocal(n\log{n})$ if not.
      \end{answer}
    
    % Put your solution here
\end{problem}

\newpage

\collaborators{None} % Put your collaborators for Problem 2 here.
\begin{problem}
    A \textit{permutation matrix} $P$ is an $n \times n$ Boolean matrix with exactly one $1$ in each row and one $1$ in each column. It is called a permutation matrix because if you multiply $P$ by any $n \times 1$ column vector $v$ then the result $Pv$ is a permutation of $v$.
    
    \begin{ppart}
    Permutation matrices are invertible. Explain how to construct the inverse of a permutation $P$ from $P$. 
    \end{ppart}
    
    % Put your solution here
    
    \begin{ppart}
    What are the possible values of the determinant of $P$? Explain why your answer is true.
    \end{ppart}
    
    % Put your solution here
    
    \begin{ppart}
    Let $P = \begin{bmatrix} 0 & 0 & 1 & 0 \\ 0 & 0 & 0 & 1 \\ 1 & 0 & 0 & 0 \\ 0 & 1 & 0 & 0 \end{bmatrix}$. Construct the inverse of $P$ and compute the determinant of $P$.
    \end{ppart}
    
    % Put your solution here
\end{problem}

\newpage

\collaborators{None} % Put your collaborators for problem 3 here

\begin{problem}

    \begin{ppart}
    Prove that the product of $2$ lower triangular matrices is also lower triangular.
    \end{ppart}
    
    % Put your solution here
    
    \begin{ppart}
    Prove that the determinant of an upper triangular matrix is the product of its diagonal elements.
    \end{ppart}
    
    % Put your solution here
    
    \begin{ppart}
    Give an example of two $3 \times 3$ triangular matrices whose product is not triangular.
    \end{ppart}
    
    % Put your solution here
    
    \begin{ppart}
    Give an example of a non-singular $3 \times 3$ matrix $M$ which has no $LU$ decomposition, i.e. $M$ is not equal to $LU$ for an $L$ and $U$ where $L$ is unit triangular and $U$ is upper triangular.
    \end{ppart}
    
    % Put your solution here

\end{problem}

\end{document}
