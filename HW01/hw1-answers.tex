\documentclass[a4paper,12pt]{article}
\usepackage[
  top=1.8cm,
  bottom=1.5cm,
  left=1.8cm,
  right=1.8cm,
  includefoot,
  includehead]{geometry}

\usepackage[utf8]{inputenc}
\usepackage{changepage}
\usepackage{multicol}
\usepackage{nccmath}
\usepackage{amsmath}
\usepackage{calc}
\usepackage{enumitem}
\usepackage[linewidth=1.2pt,linecolor=red]{mdframed}

% For fancy math
\RequirePackage{amsmath,amsthm,amssymb}
\newtheorem{theorem}{Theorem}
\newtheorem{fact}[theorem]{Fact}
\newtheorem{lemma}[theorem]{Lemma}
\newtheorem{claim}[theorem]{Claim}

\newcommand{\ord}[2][th]{\ensuremath{{#2}^{\mathrm{#1}}}}
% shorthand for \mathcal{O}
\newcommand{\Ocal}{\ensuremath{\mathcal{O}}}
\newcommand{\aug}{\fboxsep=-\fboxrule\!\!\!\fbox{\strut}\!\!\!}
\newcommand{\contradiction}{%
  \ensuremath{{\Rightarrow\mspace{-2mu}\Leftarrow}}%
}

% Counters for HW number, author, and collaborators
\newcommand{\hwnumber}[1]{\def\hwnumberdata{#1}}
\def\hwnumberdata{\relax}
\renewcommand{\author}[1]{\def\authordata{#1}}
\def\authordata{\relax}
\newcommand{\collaborators}[1]{\def\collaboratorsdata{#1}}
\def\collaboratorsdata{\relax}

% Fancy headings
\RequirePackage{fancyhdr}
\pagestyle{fancyplain}

\fancyhead[L]{\small \authordata \\
\small CS 630 Homework \#\hwnumberdata \\
  \textsl{Collaborators}: \collaboratorsdata}

\RequirePackage{titlesec}
\titleformat{\subsection}{\normalsize\bfseries}{\thesubsection}{.5em}{}
\renewcommand{\thesubsection}{\alph{subsection})}

% Making the problem and ppart environments
\newcommand{\addmedskip}{\addvspace{2\medskipamount}}
\newcommand{\addbigskip}{\addvspace{2\bigskipamount}}
\newcommand{\nline}{\bigskip}

\newcounter{problemnum}
\setcounter{problemnum}{0}
\newenvironment{problem}
  {\addbigskip \setcounter{partnum}{0}
   \noindent\stepcounter{problemnum}\textbf{Problem \arabic{problemnum}.\ }}
  {\par\addbigskip}

\newcounter{partnum}
\setcounter{partnum}{0}

\newenvironment{ppart}[1][]{%
  \addmedskip
  \refstepcounter{partnum}%\par\medskip%
  \enumerate[labelsep=*]\item[\textbf{\roman{partnum})}]}
  {\endenumerate}

\newsavebox{\mybox}
\newenvironment{answer}
{\begin{lrbox}{\mybox}\begin{minipage}{0.95\textwidth}\vspace{0.2cm}}
  {\vspace{0.1cm}\end{minipage}\end{lrbox}\fbox{\usebox{\mybox}}}

\newenvironment{customProof}{\begin{proof}\noindent}{\end{proof}}

% Put your name and the homework number here.
\author{Jiun-Yan (Eric) Chen}
\hwnumber{1}


\begin{document}
\vspace*{0.5\baselineskip}
\textbf{Please limit your answer to the following problems to at most 1/2 a page each.} 

\collaborators{None} % Put your collaborators for Problem 1 here.

\begin{problem}
    You are given an integer $t$ and a polynomial $p(x)$ of degree $n$ where all of the $n+1$ coefficient $a_i$ of $p$ are non-zero integers. You want to compute $p(t)$.
    
    \begin{ppart}
    How many integer additions and integer multiplications does it take to compute $p(t)$ the normal way, i.e. plugging in $t$ for $x$ wherever $x$ occurs in $p(x)$ and then computing each term $a_it^i$ from left to right and summing the values as you go?
    \end{ppart}
    
    \begin{answer}
      To resolve each $a_it^i$, there is a multiplication step $a_i \times t \times t \times ... \times t$ where there are $i$ $t$s
      
      So for each term, there is $i+1$ multiplication steps. As there is $i+1$ terms, the total multiplications count is:
      
      \begin{equation*}
        \sum^{i+1}_{n=1} n = \frac{i+1(i+2)}{2}
      \end{equation*}

      To sum up all terms in $p(t)$, there is $a_it^i + a_{i-1}t^{i-1} ... + a_0$, where there are $i$ additions, as there are total of $i+1$ terms.
    \end{answer}

    \begin{ppart}
    Now see if you can find a better method to compute $p(t)$, using fewer than $\Ocal(n^2)$ adds and multiplies. Describe your method and explain clearly how many arithmetic operations it uses.
    \end{ppart}

    \begin{answer}
      There are two methods that could work, the first one makes the assumption that the algorithm can implement some form of caching.
      
      If we could cache each $t^n$, then it would be a trivial $\Ocal(1)$ operation to calculate $t^{n+1}$. 
      
      Thus the multiplications for each term $a_nt^n$ is $a_n \times t^{n-1} \times t$, which is 2 operations, and the total multiplication operations is $n \times 2$.

      \nline

      If we could not cache $t^n$, then we can also use a more efficient method of obtaining the powers, namely:
      \[
        \begin{cases}
          t^n = t^{n/2} \times t^{n/2},& \text{when n is an even number}\\
          t^n = t^{(n-1)/2} \times t^{(n-1)/2} \times t, & \text{when n is an odd number}
        \end{cases}
      \]
      With this method, we would have a minimum of $\log{n}$ operations, and a maximum of $2 \times \log{n}$ for each term. Summed up, this is equal to $n \log{n}$

      \nline

      For additions, as we will always have $n + 1$ terms, we will always need to perform $n$ addition operations.

      \nline

      Thus the total complexity for calculating a polynomial is $\Ocal(2n + n) = \Ocal(n)$ if caching is allowed, or $\Ocal(n\log{n} + n) = \Ocal(n\log{n})$ if not.
      \end{answer}
    
\end{problem}

\newpage

\collaborators{None} % Put your collaborators for Problem 2 here.
\begin{problem}
    A \textit{permutation matrix} $P$ is an $n \times n$ Boolean matrix with exactly one $1$ in each row and one $1$ in each column. It is called a permutation matrix because if you multiply $P$ by any $n \times 1$ column vector $v$ then the result $Pv$ is a permutation of $v$.
    
    \begin{ppart}
    Permutation matrices are invertible. Explain how to construct the inverse of a permutation $P$ from $P$. 
    \end{ppart}
    
    \begin{answer}
      We define:
      $P = 
      \begin{bmatrix}
        a_1 & a_1 & a_1 & a_1 \\
        b_2 & b_2 & b_2 & b_2 \\
        c_3 & c_3 & c_3 & c_3 \\
        d_4 & d_4 & d_4 & d_4 
      \end{bmatrix}$, 
      $P^{-1} = 
      \begin{bmatrix}
        e_1 & f_1 & g_1 & h_1 \\
        e_2 & f_2 & g_2 & h_2 \\
        e_3 & f_3 & g_3 & h_3 \\
        e_4 & f_4 & g_4 & h_4 
      \end{bmatrix}$
      
      \nline 
      
      where for $a_i, b_i, c_i, d_i$, there is one '1' for each series $^{(1)}$, and $PP^{-1} = I$. \\      
      By matrix multiplication, we have the following
      
      \begin{equation*}
        \sum^4_{i = 1} a_i \times e_i = 1, \qquad
        \sum^4_{i = 1} b_i \times f_i = 1, \qquad
        \sum^4_{i = 1} c_i \times g_i = 1, \qquad
        \sum^4_{i = 1} d_i \times h_i = 1
      \end{equation*}\\
      and all other series combinations being 0, then because of $^{(1)}$, $a_i = e_i, b_i = f_i, c_i = g_i, d_i = h_i$. Thus, $P^{-1}$, or the inverse of $P$, is naturally also the transpose of $P$.
    \end{answer}
    
    \begin{ppart}
    What are the possible values of the determinant of $P$? Explain why your answer is true.
    \end{ppart}

    \begin{answer}
      The only possible values are $1$ and $-1$. Since the only elementary row operations needed to transform a permutation matrix $P$ into an indentity matrix $I$ is the row swap, and no row addition or multiplication is needed, the determinant of $P$ must be equal to the determinant of $I$ (which is equal to 1), times the amount $-1^i$, where $i$ is the amount of row swaps needed.
    \end{answer}
    
    \bigskip
    
    \begin{ppart}
    Let $P = \begin{bmatrix} 0 & 0 & 1 & 0 \\ 0 & 0 & 0 & 1 \\ 1 & 0 & 0 & 0 \\ 0 & 1 & 0 & 0 \end{bmatrix}$. Construct the inverse of $P$ and compute the determinant of $P$.
    \end{ppart}

    \begin{answer}
        The inverse of $P$: $P^{-1} = P^T = 
        \begin{bmatrix}
          0 & 0 & 1 & 0\\
          0 & 0 & 0 & 1\\
          1 & 0 & 0 & 0\\
          0 & 1 & 0 & 0\\
        \end{bmatrix}$

      \nline

      The following steps is needed to transform $P$ into $I$: $r_1 \leftrightarrow r_3, r_2 \leftrightarrow r_4$\\
      Thus, the determinant of $P$ is $-1^2 = 1$
    \end{answer}

\end{problem}

\newpage

\collaborators{None} % Put your collaborators for problem 3 here

\begin{problem}

    \begin{ppart}
    Prove that the product of $2$ lower triangular matrices is also lower triangular.
    \end{ppart}
    
    \begin{answer}
      \begin{customProof}
        Let $A$, $B$, be a lower triangular matrix of size $k \times x$ with elements $a_{i,j}/b_{i,j}$, $C = AB$
          
          \begin{align*}
            c_{i,j} &= \sum_1^k a_{i,n} b_{n,j} \\
            &= \sum_1^i a_{i,n} b_{n,j} & (a_{i,n} = 0 \textrm{ when $i < n$})\\
            &= \sum_j^i a_{i,n} b_{n,j} & (b_{n,j} = 0 \textrm{ when $n < j$})
          \end{align*}
          
          $\therefore c_{i,j} = 0$ when $i < j$, $\implies C$ must also be a lower triangular matrix.
        \end{customProof}
    \end{answer}
    
    \begin{ppart}
    Prove that the determinant of an upper triangular matrix is the product of its diagonal elements.
    \end{ppart}
    
    \begin{answer}
      \begin{customProof}
        Let $A$ be an upper triangular matrix,$B$ be a diagonal matrix that can be made by performing row additions on $A$. By definition, $B$ must exist, and $det(B) = $ {\it the product of it's diagonal elements}, therefore $det(A) = det(B) = $ {\it product of it's diagonal elements.}
      \end{customProof}
    \end{answer}
      
    \begin{ppart}
    Give an example of two $3 \times 3$ triangular matrices whose product is not triangular.
    \end{ppart}
    
    \begin{answer}
      Let $A = \begin{bmatrix}
        1 & 1 & 1 \\
        0 & 1 & 1 \\
        0 & 0 & 1
      \end{bmatrix}$
      $B = \begin{bmatrix}
        1 & 0 & 0 \\
        1 & 1 & 0 \\
        1 & 1 & 1
      \end{bmatrix}$
      $AB = \begin{bmatrix}
        3 & 2 & 1 \\
        2 & 2 & 1 \\
        1 & 1 & 1
      \end{bmatrix}$, 
      
      \nline

      $\implies AB$ is not a triangular matrix.
    \end{answer}
    
    \begin{ppart}
    Give an example of a non-singular $3 \times 3$ matrix $M$ which has no $LU$ decomposition, i.e. $M$ is not equal to $LU$ for an $L$ and $U$ where $L$ is unit triangular and $U$ is upper triangular.
    \end{ppart}
    
    \begin{answer}
      \begin{customProof}
        Let $A = \begin{bmatrix}
          0 & 0 & 1 \\
        0 & 1 & 1 \\
        1 & 1 & 1
      \end{bmatrix}$, supposed $A$ has an $LU$ decomposition, 
      
      \nline

      where 
      $L = \begin{bmatrix}
        1 & 0 & 0 \\
        l_{2,1} & 1 & 0 \\
        l_{3,1} & l_{3,2} & 1 \\
      \end{bmatrix}$,
      $U = \begin{bmatrix}
        u_{1,1} & u_{1,2} & u_{1,3} \\
        0 & u_{2,2} & u_{2,3} \\
        0 & 0 & u_{3,3} \\
      \end{bmatrix}$

      \nline
      
      Since $0 = \sum_{i=1}^3 l_{1,i} u_{i,1} = u_{1,1}$, and $1 = \sum_{i=3}^3 l_{3,i} u_{i,1} = l_{3,1} u_{1,1} = 0 \contradiction, \\\therefore A$ does not have an $LU$ decomposition where L is a unit lower triangular
    \end{customProof}
  \end{answer}
    
\end{problem}

\end{document}
